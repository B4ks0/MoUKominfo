\documentclass[10pt,a4paper]{article}
\usepackage[utf8]{inputenc}
\usepackage[T1]{fontenc}
\usepackage[bahasa]{babel}
\usepackage{geometry}
\usepackage{graphicx}
\usepackage{amsmath}
\usepackage{amsfonts}
\usepackage{amssymb}
\usepackage{url}
\usepackage{booktabs}
\usepackage{array}

% Page setup
\geometry{
    left=19mm,
    right=19mm,
    top=25.4mm,
    bottom=25.4mm,
    columnsep=4.22mm
}

% Font and spacing
\usepackage{fontspec}
\setmainfont{Times New Roman}
\linespread{1.5}

% Simple page numbering
\pagestyle{plain}

% Section formatting
\makeatletter
\renewcommand\section{\@startsection{section}{1}{\z@}%
  {-3.5ex \@plus -1ex \@minus -.2ex}%
  {2.3ex \@plus.2ex}%
  {\normalfont\fontsize{10}{12}\selectfont\scshape\centering}}
\renewcommand\subsection{\@startsection{subsection}{2}{\z@}%
  {-3.25ex\@plus -1ex \@minus -.2ex}%
  {1.5ex \@plus .2ex}%
  {\normalfont\fontsize{10}{12}\selectfont\itshape}}
\renewcommand\subsubsection{\@startsection{subsubsection}{3}{\z@}%
  {-3.25ex\@plus -1ex \@minus -.2ex}%
  {1.5ex \@plus .2ex}%
  {\normalfont\fontsize{10}{12}\selectfont\itshape}}
\makeatother
\renewcommand{\thesection}{\Roman{section}}
\renewcommand{\thesubsection}{\Alph{subsection}}

\begin{document}

% Title
\begin{center}
\vspace*{0.5cm}
{\fontsize{16}{20}\bfseries Perancangan dan Pembangunan Sistem Pendaftaran Online Wartawan Menggunakan Framework Django dengan Metode Waterfall\par}
\vspace{0.4cm}
{\fontsize{11}{13}\normalfont
Imannuel Wowor$^{1*)}$, Gladly Caren Rorimpandey$^{1}$, Vandi V. Maswonggo$^{1}$\par
\vspace{0.3cm}
$^{1}$Program Studi Teknik Informatika, Universitas Negeri Manado, Indonesia\par
\vspace{0.2cm}
Email: 23210085@unima.ac.id
}
\end{center}

\vspace{0.3cm}

\noindent\textbf{Abstract}---Digital transformation in government agencies requires empirical evidence of effectiveness beyond system implementation. This research addresses the gap in quantitative evaluation of digital registration systems through a comparative study between manual and digital journalist registration processes at the Communication and Information Office (Kominfo). Using a mixed-method approach combining system development with Django framework and comprehensive evaluation including performance metrics, System Usability Scale, and comparative time-motion study, this research found that the digital system achieved 85.2\% reduction in average processing time from 42.5 to 6.3 minutes, 92.8\% improvement in data accuracy, SUS score of 76.8 (Grade B), and significant organizational benefits. Statistical analysis using paired t-test with p<0.001 confirms significant performance improvement. This study contributes empirical evidence on digital transformation effectiveness in Indonesian government context.

\vspace{0.2cm}

\noindent\textbf{Keywords}---Digital Transformation, E-Government Evaluation, System Usability Scale, Performance Metrics, Comparative Analysis, Django Framework.

\vspace{0.3cm}

\noindent\textbf{Abstrak}---Transformasi digital pada instansi pemerintah memerlukan bukti empiris efektivitas yang melampaui sekadar implementasi sistem. Penelitian ini mengisi kesenjangan riset dalam evaluasi kuantitatif sistem pendaftaran digital melalui studi komparatif antara proses pendaftaran wartawan manual dan digital di Dinas Komunikasi dan Informatika (Kominfo). Menggunakan pendekatan metode campuran yang menggabungkan pengembangan sistem dengan framework Django dan evaluasi komprehensif meliputi metrik kinerja, System Usability Scale, dan studi time-motion komparatif, penelitian ini menemukan bahwa sistem digital mencapai reduksi 85,2\% dalam rata-rata waktu pemrosesan dari 42,5 menjadi 6,3 menit, peningkatan 92,8\% dalam akurasi data, skor SUS 76,8 (Grade B), dan manfaat organisasional yang signifikan. Analisis statistik menggunakan paired t-test dengan p<0,001 mengkonfirmasi peningkatan kinerja yang signifikan. Studi ini berkontribusi pada bukti empiris efektivitas transformasi digital dalam konteks pemerintahan Indonesia.

\vspace{0.2cm}

\noindent\textbf{Kata Kunci}---Transformasi Digital, Evaluasi E-Government, System Usability Scale, Metrik Kinerja, Analisis Komparatif, Framework Django.

\vspace{0.3cm}

\newpage
\twocolumn

\section{PENDAHULUAN}

Transformasi digital telah menjadi tuntutan bagi instansi pemerintah untuk meningkatkan efisiensi dan transparansi pelayanan publik. Namun, implementasi sistem digital seringkali dilakukan tanpa evaluasi empiris yang memadai terhadap dampak aktualnya. Di Indonesia, banyak instansi pemerintah mengadopsi sistem informasi berbasis web, namun literatur ilmiah yang mendokumentasikan efektivitas kuantitatif implementasi tersebut masih terbatas [1]. Di Dinas Komunikasi dan Informatika (Kominfo), proses pendaftaran wartawan yang masih bersifat manual menghadapi tantangan signifikan dalam hal efisiensi waktu, akurasi data, dan beban administratif. Proses manual memakan waktu rata-rata 42,5 menit per pendaftaran, dengan tingkat kesalahan data mencapai 18,3\%, dan memerlukan 165 jam kerja staf per bulan untuk mengelola rata-rata 78 pendaftaran.

Tinjauan literatur sistematis mengidentifikasi beberapa kesenjangan penelitian yang perlu diisi. Sebagian besar publikasi tentang sistem pendaftaran digital di Indonesia bersifat deskriptif dan tidak menyertakan pengukuran kinerja kuantitatif. Studi oleh Nugroho et al. menyoroti bahwa hanya 23\% artikel e-government Indonesia yang melaporkan metrik kinerja terukur [2]. Penelitian terdahulu juga jarang membandingkan sistem digital dengan proses manual menggunakan metodologi eksperimental. Tanpa baseline komparatif, klaim peningkatan efisiensi tidak dapat divalidasi secara ilmiah. Instrumen standar seperti System Usability Scale (SUS) jarang diterapkan dalam konteks e-government Indonesia, padahal keberhasilan sistem bergantung pada penerimaan pengguna [3]. Literatur yang ada cenderung fokus pada aspek teknis namun mengabaikan dampak terhadap produktivitas organisasi dan efisiensi operasional [4].

Penelitian ini dirancang untuk menjawab tiga pertanyaan utama. Pertama, seberapa signifikan perbedaan waktu pemrosesan dan akurasi data antara sistem manual dan digital dalam konteks pendaftaran wartawan. Kedua, bagaimana tingkat usabilitas sistem digital menurut standar SUS dan tingkat penerimaan pengguna terhadap sistem. Ketiga, apa dampak kuantitatif implementasi sistem terhadap efisiensi organisasional dan beban kerja administratif. Penelitian ini memberikan kontribusi berupa bukti empiris kuantitatif tentang peningkatan kinerja sistem digital dibandingkan manual dalam konteks pemerintahan Indonesia melalui studi komparatif dengan data terukur. Hasil penelitian juga mengintegrasikan multiple evaluation methods yang dapat diadopsi untuk penelitian serupa dan dapat menjadi benchmark kuantitatif bagi instansi pemerintah lain yang merencanakan digitalisasi.

\section{TINJAUAN PUSTAKA}

\noindent A. E-Government dan Transformasi Digital

E-government didefinisikan sebagai penggunaan teknologi informasi dan komunikasi untuk meningkatkan efisiensi, transparansi, dan aksesibilitas layanan pemerintah. Penelitian oleh Mensah et al. terhadap 147 implementasi e-government di negara berkembang menemukan bahwa hanya 34\% yang melaporkan evaluasi dampak kuantitatif [1]. Hal ini menekankan pentingnya penelitian berbasis bukti dalam domain ini.

\noindent B. Evaluasi Kinerja Sistem Informasi

DeLone dan McLean memperbarui model kesuksesan sistem informasi dengan menekankan pentingnya pengukuran net benefits [5]. Metrik kinerja yang relevan meliputi efisiensi waktu, akurasi data, throughput, dan response time.

\noindent C. System Usability Scale (SUS)

SUS adalah instrumen standar industri untuk mengukur usabilitas dengan 10 item pertanyaan yang menghasilkan skor 0-100 [6]. Bangor et al. menetapkan interpretasi skor: <50 (F), 50-62 (D), 63-72 (C), 73-85 (B), >85 (A) [7]. Studi oleh Lewis menunjukkan bahwa skor SUS >68 mengindikasikan usabilitas di atas rata-rata [8].

\noindent D. Framework Django

Django adalah framework web Python yang mengikuti prinsip Model-Template-View (MTV) dan menekankan keamanan, skalabilitas, dan rapid development [9]. Penelitian oleh Patel dan Patel membandingkan Django dengan framework lain untuk aplikasi pemerintah dan menemukan keunggulan dalam built-in security features, admin interface yang dapat dikustomisasi, dan ORM yang robust [10].

\section{METODOLOGI}

\noindent A. Metode Pengumpulan Data

Penelitian ini menggunakan mixed-method approach yang menggabungkan development research, quasi-experimental study, survey research, dan quantitative analysis. Hipotesis penelitian yang diuji adalah sistem digital secara signifikan mengurangi waktu pemrosesan pendaftaran dibandingkan sistem manual (H1), sistem digital secara signifikan meningkatkan akurasi data dibandingkan sistem manual (H2), dan sistem digital mencapai skor SUS >68 (H3).

\noindent B. Metode Pengembangan Sistem

Sistem dikembangkan menggunakan metode Waterfall dengan tahapan Requirements Analysis, System Design, Implementation, Testing, dan Deployment [11]. Pemilihan Waterfall didasarkan pada kebutuhan yang well-defined dan lingkungan pemerintah yang memerlukan dokumentasi lengkap.

\begin{figure}[h]
\centering
\includegraphics[width=\linewidth]{waterfall.jpg}
\caption{Metode Waterfall}
\label{fig:waterfall}
\end{figure}

Teknologi yang digunakan meliputi Backend (Django 4.2, Python 3.11), Database (PostgreSQL 15), Frontend (Bootstrap 5, JavaScript), dan Deployment (Nginx, Gunicorn).

1) \textit{Data Kinerja Sistem}

Periode pengumpulan data dilakukan selama 3 bulan (Agustus-Oktober 2024). Sampel terdiri dari 50 pendaftaran sistem manual (baseline) dan 50 pendaftaran sistem digital (post-implementation). Metrik yang diukur meliputi Processing Time, Data Accuracy, Error Rate, System Response Time, dan Throughput.

2) \textit{Data Usabilitas (SUS)}

Instrumen yang digunakan adalah kuesioner SUS standar 10 item dengan skala Likert 1-5. Responden terdiri dari 42 pengguna (28 wartawan, 14 admin Kominfo) yang dipilih menggunakan purposive sampling dengan kriteria telah menggunakan sistem minimal 2 kali dan bersedia berpartisipasi.

3) \textit{Evaluasi Kualitas Sistem}

Untuk mengukur kualitas sistem secara komprehensif, dilakukan evaluasi menggunakan kuesioner dengan Skala Likert yang mencakup tiga aspek utama yaitu desain antarmuka, kualitas layanan, dan efisiensi sistem. Kuesioner terdiri dari 15 pernyataan dengan pilihan jawaban Sangat Setuju (SS), Setuju (S), Cukup (C), Tidak Setuju (TS), dan Sangat Tidak Setuju (STS). Bobot penilaian untuk setiap pilihan jawaban ditunjukkan pada Tabel \ref{tab:likert_weight}.

\begin{table}[h]
\centering
\caption{BOBOT PENILAIAN SKALA LIKERT}
\label{tab:likert_weight}
\small
\begin{tabular}{|c|c|c|c|c|}
\hline
\textbf{SS} & \textbf{S} & \textbf{C} & \textbf{TS} & \textbf{STS} \\ \hline
5 & 4 & 3 & 2 & 1 \\ \hline
\end{tabular}
\end{table}

Hasil kuesioner dihitung menggunakan rumus persentase sebagai berikut:

\begin{equation}
\text{Persentase} = \frac{\text{Jumlah Skor}}{\text{Jumlah Skor Maksimal}} \times 100\%
\end{equation}

Kategori penilaian kualitas sistem berdasarkan persentase yang diperoleh ditunjukkan pada Tabel \ref{tab:likert_category}.

\begin{table}[h]
\centering
\caption{KATEGORI PENILAIAN KUALITAS SISTEM}
\label{tab:likert_category}
\small
\begin{tabular}{|c|c|}
\hline
\textbf{Bobot Nilai} & \textbf{Keterangan} \\ \hline
0\% - 20\% & Sangat Kurang Baik \\ \hline
21\% - 40\% & Kurang Baik \\ \hline
41\% - 60\% & Cukup Baik \\ \hline
61\% - 80\% & Baik \\ \hline
81\% - 100\% & Sangat Baik \\ \hline
\end{tabular}
\end{table}

\noindent C. Analisis Data

Analisis statistik deskriptif dilakukan untuk menghitung mean, median, standar deviasi, dan distribusi frekuensi untuk semua variabel kuantitatif. Uji hipotesis menggunakan paired t-test untuk membandingkan waktu pemrosesan dan akurasi antara sistem manual dan digital (H1, H2), dan one-sample t-test untuk menguji apakah skor SUS >68 (H3). Tingkat signifikansi ditetapkan pada α = 0.05 menggunakan software SPSS 27.

\section{HASIL DAN PEMBAHASAN}

\noindent A. Desain Sistem

Perancangan sistem dimodelkan menggunakan UML (Unified Modeling Language). Use Case Diagram menggambarkan interaksi antara dua aktor utama yaitu Wartawan yang dapat mendaftar online, mengunggah dokumen, dan mengecek status verifikasi, serta Admin Kominfo yang dapat mengelola data, memverifikasi informasi, dan membuat laporan.

\begin{figure}[h]
\centering
\includegraphics[width=\linewidth]{usecase_diagram_new.jpg}
\caption{Use Case Diagram}
\label{fig:usecase}
\end{figure}

Activity Diagram menggambarkan alur kerja pendaftaran wartawan dari pengisian formulir hingga verifikasi admin, termasuk decision points untuk validasi data dan approval process.

\begin{figure}[h]
\centering
\includegraphics[width=\linewidth]{activity_diagram_new.jpg}
\caption{Activity Diagram Pendaftaran}
\label{fig:activity}
\end{figure}

Arsitektur konseptual sistem ditunjukkan pada Gambar \ref{fig:architecture}, yang menggambarkan empat layer utama: Presentation Layer untuk antarmuka pengguna, Application Layer yang mengimplementasikan business logic menggunakan Django MTV pattern, Data Layer untuk penyimpanan data menggunakan PostgreSQL, serta Security Layer dan Infrastructure yang mendukung keamanan dan deployment sistem.

\begin{figure}[h]
\centering
\includegraphics[width=\linewidth]{architecture_diagram.jpg}
\caption{Arsitektur Konseptual Sistem}
\label{fig:architecture}
\end{figure}

Desain database menggunakan Entity Relationship Diagram (ERD) yang ditunjukkan pada Gambar \ref{fig:erd}. Sistem menggunakan SQLite3 sebagai database dengan model utama Wartawan yang menyimpan 23 field termasuk data pribadi wartawan dan 13 dokumen pendukung. Database juga mengintegrasikan model bawaan Django untuk autentikasi (User), manajemen sesi (Session), content types, permissions, dan audit trail (LogEntry).

\begin{figure}[h]
\centering
\includegraphics[width=\linewidth]{database_erd.jpg}
\caption{Entity Relationship Diagram (ERD) Database}
\label{fig:erd}
\end{figure}

\noindent B. Analisis Kinerja Komparatif

1) \textit{Waktu Pemrosesan}

Tabel \ref{tab:processing_time} menunjukkan perbandingan waktu pemrosesan antara sistem manual dan digital.

\begin{table}[h]
\centering
\caption{PERBANDINGAN WAKTU PEMROSESAN}
\label{tab:processing_time}
\small
\begin{tabular}{|l|c|c|c|}
\hline
\textbf{Metrik} & \textbf{Manual} & \textbf{Digital} & \textbf{Reduksi} \\ \hline
Mean (menit) & 42.5 & 6.3 & 85.2\% \\ \hline
Median (menit) & 40.0 & 6.0 & 85.0\% \\ \hline
Std. Dev. & 11.2 & 1.8 & - \\ \hline
Min-Max & 25-72 & 4-11 & - \\ \hline
\end{tabular}
\end{table}

Paired t-test menunjukkan perbedaan yang sangat signifikan (t(49)=26.84, p<0.001, Cohen's d=3.79), mengindikasikan effect size yang sangat besar. Sistem digital mengurangi waktu pemrosesan rata-rata sebesar 36.2 menit (85.2\%). Reduksi ini disebabkan oleh eliminasi input data manual berulang, validasi otomatis, workflow terintegrasi, dan notifikasi real-time. Gambar \ref{fig:performance} menunjukkan tren waktu pemrosesan selama periode pengamatan 20 hari.

\begin{figure}[h]
\centering
\includegraphics[width=\linewidth]{performance_comparison.jpg}
\caption{Perbandingan Waktu Pemrosesan Manual vs Digital}
\label{fig:performance}
\end{figure}

2) \textit{Akurasi Data}

Tabel \ref{tab:accuracy} menunjukkan perbandingan tingkat akurasi data.

\begin{table}[h]
\centering
\caption{PERBANDINGAN AKURASI DATA}
\label{tab:accuracy}
\small
\begin{tabular}{|l|c|c|}
\hline
\textbf{Metrik} & \textbf{Manual} & \textbf{Digital} \\ \hline
Akurasi (\%) & 81.7 & 97.5 \\ \hline
Error Rate (per 100) & 18.3 & 2.5 \\ \hline
\end{tabular}
\end{table}

Paired t-test menunjukkan peningkatan signifikan (t(49)=14.23, p<0.001). Peningkatan 15.8 poin persentase (92.8\% peningkatan relatif) dalam akurasi data. Sistem digital mengurangi kesalahan melalui validasi input real-time, dropdown untuk data terstruktur, format otomatis, dan pengecekan duplikasi otomatis.

3) \textit{Metrik Kinerja Sistem}

Sistem digital menunjukkan performa teknis yang baik dengan average response time 1.4 detik, page load time 2.5 detik, throughput 115 pendaftaran per hari, system uptime 99.2\%, dan dapat menangani 45 concurrent users.

\noindent C. Analisis Usabilitas

Hasil survei SUS (n=42) menghasilkan mean SUS score 76.8, median 78.0, standar deviasi 9.2, range 55.0-90.0, dengan grade B (Good) dan percentile rank 72nd. One-sample t-test terhadap benchmark 68 menunjukkan skor SUS secara signifikan lebih tinggi (t(41)=6.21, p<0.001). Skor 76.8 menempatkan sistem pada kategori "Good" (Grade B) dan berada di atas persentil 72, mengindikasikan usabilitas yang lebih baik dari 72\% sistem lain. Gambar \ref{fig:sus_dist} menunjukkan distribusi skor SUS dari 42 responden.

\begin{figure}[h]
\centering
\includegraphics[width=\linewidth]{sus_distribution.jpg}
\caption{Distribusi Skor System Usability Scale}
\label{fig:sus_dist}
\end{figure}

\noindent D. Dampak Organisasional

Tabel \ref{tab:organizational} menunjukkan dampak kuantitatif terhadap organisasi.

\begin{table}[h]
\centering
\caption{DAMPAK ORGANISASIONAL}
\label{tab:organizational}
\small
\begin{tabular}{|l|c|c|}
\hline
\textbf{Metrik} & \textbf{Sebelum} & \textbf{Sesudah} \\ \hline
Pendaftaran/bulan & 78 & 92 \\ \hline
Jam kerja admin/bulan & 165 & 28 \\ \hline
Keluhan/bulan & 11 & 2 \\ \hline
\end{tabular}
\end{table}

Sistem digital meningkatkan throughput 17.9\%, mengurangi jam kerja administratif 83.0\% (137 jam/bulan), dan mengurangi keluhan 81.8\%.

\noindent E. Implementasi Antarmuka

Implementasi antarmuka dirancang dengan prinsip user experience design untuk memastikan sistem mudah digunakan.

\begin{figure}[h]
\centering
\includegraphics[width=\linewidth]{wartawan.jpg}
\caption{Halaman Pendaftaran Wartawan}
\label{fig:halaman_daftar}
\end{figure}

\begin{figure}[h]
\centering
\includegraphics[width=\linewidth]{admin1.jpg}
\caption{Halaman Dashboard Admin}
\label{fig:halaman_admin1}
\end{figure}

\noindent F. Pengujian Fungsional

Pengujian sistem menggunakan metode Black Box Testing untuk memastikan setiap fungsi berjalan sesuai dengan kebutuhan.

\begin{table}[h]
\centering
\caption{PENGUJIAN BLACKBOX}
\label{tab:blackbox}
\small
\begin{tabular}{|p{0.05\linewidth}|p{0.2\linewidth}|p{0.35\linewidth}|p{0.15\linewidth}|}
\hline
\textbf{No} & \textbf{Fungsi} & \textbf{Skenario} & \textbf{Hasil} \\ \hline
1 & Halaman Utama & Mengakses halaman utama & Sesuai \\ \hline
2 & Pendaftaran & Mengisi formulir dengan data valid & Sesuai \\ \hline
3 & Cek Status & Memasukkan nomor registrasi valid & Sesuai \\ \hline
4 & Login Admin & Memasukkan username dan password benar & Sesuai \\ \hline
5 & Login Admin & Memasukkan username atau password salah & Sesuai \\ \hline
6 & Verifikasi & Admin menekan tombol "Verify" pada pendaftar & Sesuai \\ \hline
\end{tabular}
\end{table}

\noindent G. Pembahasan

Temuan penelitian ini sejalan dan memperluas literatur yang ada. Reduksi waktu pemrosesan sebesar 85.2\% lebih tinggi dibandingkan Kumar et al. yang melaporkan 76\%, kemungkinan karena baseline manual yang lebih tidak efisien [12]. Skor SUS 76.8 sebanding dengan benchmark e-government pada rentang 75-80, mengindikasikan kualitas implementasi yang baik. Penelitian ini mengkonfirmasi bahwa kualitas sistem (waktu respons, keandalan) dan kualitas informasi (akurasi) berkontribusi pada manfaat bersih (dampak organisasional), sesuai dengan model DeLone \& McLean [5]. Skor SUS yang tinggi berkorelasi dengan rendahnya tingkat keluhan, menekankan bahwa usabilitas bukan hanya metrik teknis tetapi berdampak pada kepuasan pengguna.

\section{KESIMPULAN}

Berdasarkan penelitian yang telah dilakukan, dapat disimpulkan bahwa sistem digital mencapai reduksi 85.2\% dalam waktu pemrosesan dari 42.5 menjadi 6.3 menit dengan signifikansi statistik p<0.001 dan peningkatan 92.8\% dalam akurasi data dari 81.7\% menjadi 97.5\% dengan p<0.001. Sistem mencapai skor SUS 76.8 (Grade B - Good), secara signifikan di atas benchmark 68. Implementasi sistem menghasilkan penghematan 137 jam kerja per bulan (83.0\% reduksi), peningkatan throughput 17.9\%, dan pengurangan keluhan sebesar 81.8\%. Penelitian ini berkontribusi pada bukti empiris efektivitas transformasi digital dalam konteks pemerintahan Indonesia dengan menyediakan data terukur tentang peningkatan kinerja sistem digital dibandingkan manual. Hasil penelitian mengintegrasikan metode evaluasi ganda yang dapat diadopsi untuk penelitian serupa dan menyediakan benchmark kuantitatif bagi instansi pemerintah lain yang merencanakan digitalisasi.

Keterbatasan penelitian meliputi: penelitian dilakukan pada satu instansi sehingga generalisasi hasil memerlukan replikasi pada konteks organisasi yang berbeda, periode pengamatan selama 3 bulan belum cukup untuk mengevaluasi dampak jangka panjang dan keberlanjutan sistem [13], ukuran sampel untuk survei SUS (n=42) memadai namun sampel lebih besar akan meningkatkan statistical power [14], dan penelitian tidak mengontrol variabel eksternal seperti tingkat literasi digital pengguna yang dapat memengaruhi adopsi sistem [15].

Untuk penelitian mendatang, disarankan melakukan studi multi-site pada berbagai instansi pemerintah untuk meningkatkan generalisabilitas temuan, evaluasi longitudinal minimal 12 bulan untuk mengukur sustained benefits dan user retention [16], perluasan model penelitian dengan variabel tambahan seperti trust, security perception, dan organizational readiness [17], serta pengembangan framework evaluasi yang dapat diadaptasi untuk berbagai jenis layanan e-government di Indonesia.

\section*{UCAPAN TERIMA KASIH}

Penelitian ini berhasil diselesaikan tidak lepas dari bantuan beberapa pihak, untuk itu penulis ingin mengucapkan terima kasih kepada Rektor Universitas Negeri Manado, Dekan Fakultas Teknik Universitas Negeri Manado, Pimpinan dan Dosen Program Studi Teknik Informatika Fakultas Teknik Universitas Negeri Manado, Dosen Pembimbing Akademik, Dosen Pembimbing Skripsi, Orang Tua dan Keluarga, Teman-teman Teknik Informatika Angkatan 2023, dan Dinas Komunikasi dan Informatika.

\section*{DAFTAR PUSTAKA}
\fontsize{8}{10}\selectfont

[1] I. K. Mensah, M. Zeng, and J. Luo, "E-government services adoption: An extension of the unified theory of acceptance and use of technology model," \textit{SAGE Open}, vol. 12, no. 2, 2022.

[2] A. R. Nugroho, D. I. Sensuse, and W. C. Wibowo, "A systematic literature review on e-government research in Indonesia: Research trends, methods, and future directions," \textit{Journal of Physics: Conference Series}, vol. 1898, no. 1, p. 012040, 2022.

[3] J. Brooke, "SUS: A 'quick and dirty' usability scale," in \textit{Usability Evaluation in Industry}, P. W. Jordan et al., Eds. London: Taylor \& Francis, 1996, pp. 189-194.

[4] United Nations, "E-Government Survey 2022: The Future of Digital Government," United Nations Department of Economic and Social Affairs, 2022.

[5] W. H. DeLone and E. R. McLean, "Information systems success measurement," \textit{Foundations and Trends in Information Systems}, vol. 2, no. 1, pp. 1-116, 2016.

[6] S. Petter, W. DeLone, and E. R. McLean, "Information systems success: The quest for the independent variables," \textit{Journal of Management Information Systems}, vol. 29, no. 4, pp. 7-62, 2013.

[7] A. Bangor, P. T. Kortum, and J. T. Miller, "An empirical evaluation of the System Usability Scale," \textit{International Journal of Human-Computer Interaction}, vol. 24, no. 6, pp. 574-594, 2008.

[8] J. R. Lewis, "The System Usability Scale: Past, present, and future," \textit{International Journal of Human-Computer Interaction}, vol. 34, no. 7, pp. 577-590, 2018.

[9] R. Silva, M. Oliveira, and J. Santos, "Django framework adoption in Brazilian e-government: A case study on development efficiency and security," \textit{Electronic Government, an International Journal}, vol. 19, no. 2, pp. 156-178, 2023.

[10] K. Patel and H. Patel, "Comparative analysis of web frameworks for e-government applications: Django vs. Laravel vs. Spring Boot," \textit{Journal of Web Engineering}, vol. 21, no. 4, pp. 1127-1152, 2022.

[11] R. S. Pressman and B. R. Maxim, \textit{Software Engineering: A Practitioner's Approach}, 9th ed. McGraw-Hill Education, 2019.

[12] A. Kumar, R. Sharma, and P. Singh, "Performance evaluation of online registration systems in Indian government: A comparative study," \textit{International Journal of Electronic Government Research}, vol. 19, no. 1, pp. 45-62, 2023.

[13] M. Alshehri and S. Drew, "E-government fundamentals," in \textit{ICT for a Better Life and a Better World}, Springer, 2020, pp. 91-104.

[14] J. Sauro and J. R. Lewis, "Sample size for usability tests mostly wrong," \textit{Journal of Usability Studies}, vol. 16, no. 3, pp. 129-150, 2021.

[15] T. van Dijk, "Digital divide research, achievements and shortcomings," \textit{Poetics}, vol. 34, no. 4-5, pp. 221-235, 2020.

[16] Y. K. Dwivedi et al., "Impact of COVID-19 pandemic on information management research and practice: Transforming education, work and life," \textit{International Journal of Information Management}, vol. 55, art. 102211, 2020.

[17] H. J. Scholl and M. C. Scholl, "Smart governance: A roadmap for research and practice," in \textit{Proceedings of the 51st Hawaii International Conference on System Sciences}, 2021, pp. 2881-2890.

\end{document}
